\chapter{Introduction}
\label{ch:Introduction}%

\section{Purpose}
\label{sec:Purpose}%

\par The primary purpose of this document is to provide a comprehensive blueprint that outlines how the software system
will be developed. It acts as detailed road-map that guides the developers through the entire implementation process.

\par Also, the following text will ensure that everyone involved in the project has a shared understanding of the
project's technical approach thus helping align expectations across the various teams and reducing misunderstandings.

\par The document captures and justifies key architectural decisions, explaining why certain technologies, patterns, or
approaches were chosen. This helps maintain design consistency and provides rationale for future reference.

\section{Scope}
\label{sec:Scope}%

\par As described in the Requirements Analysis and Specification Document (RASD), S\&C is a web application that aims
to connect Students and Companies in order to facilitate the recruitment process. Students can find internships that
match their skills and interests more easily, while Companies can find the best candidates for their positions.

\par The end goal is to allow students to gain practical experience in their field of study, thus reducing the skill
mismatch between the academic and the professional world, while also providing companies with a pool of talented
candidates to choose from when hiring.

\section{Definitions, Acronyms, and Abbreviations}
\label{sec:definitions-acronyms-abbreviations}%

\begin{longtable}{|l|p{0.835\textwidth}|}
      \hline
      \textbf{Acronym} & \textbf{Definition}                               \\
      \hline \hline
      DD               & Design Document.                                  \\
      \hline
      RADS             & Requirements Analysis and Specification Document. \\
      \hline
      S\&C             & Students\&Companies.                              \\
      \hline
      ST               & Student.                                          \\
      \hline
      CO               & Company.                                          \\
      \hline
      UN               & University.                                       \\
      \hline
      CV               & Curriculum Vitae.                                 \\
      \hline
      \caption{Definitions, Acronyms, and Abbreviations Table}
      \label{tab:definitions-acronyms-abbreviations}
\end{longtable}

\section{Revision History}
\label{sec:revision-history}%

\begin{itemize}
      %% FIXME: Update the revision history before the final release.
      \item \textbf{Version 1.0} (2024-12-01): Initial release.
\end{itemize}

\section{Reference Documents}
\label{sec:reference-documents}%

\begin{itemize}
      \item Assignment description.
      \item Provided course materials and notes.
\end{itemize}

\section{Document Structure}
\label{sec:document-structure}%

\begin{enumerate}
      \item \textbf{Introduction}: This section serves as the entry point to the document, providing a concise overview
            of the document's purpose, scope, key definitions, revision history, and reference materials. It provides a
            clear context for understanding the following technical details.
      \item \textbf{Architectural Design}: This section presents the system's technical blueprint, detailing high-level
            components, their interactions, deployment strategies, and runtime behaviors. It explains the architectural
            choices and design intentions that shaped the system's structure.
      \item \textbf{User Interface Design}: This section describes the system's user interface design, illustrating how
            users will interact with the system. It provides visual and functional insights into the interface's layout,
            navigation, and interaction models.
      \item \textbf{Requirements Traceability}: This section maps requirements from the RASD to specific design elements,
            demonstrating how each system requirement is addressed and implemented through architectural and design
            choices.
            \pagebreak
      \item \textbf{Implementation, Integration, and Test Plan}: This section outlines how system components will be
            validated. It details the planned sequence of implementation, integration, and testing to ensure the desired
            system qualities.
      \item \textbf{Effort Spent}: This section documents the hours worked by each team member.
      \item \textbf{References}: This section collects all sources cited within the document.
\end{enumerate}
