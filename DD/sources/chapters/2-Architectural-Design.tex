\chapter{Architectural Design}
\label{ch:architectural-design}%

% This should be section 2.6. of the DD.

\section{Selected Architectural Styles and Patterns}
\label{sec:selected-architectural-styles-patterns}%

\par{\textbf{Architectural Design:}} The architectural design of S\&C is based on the industry-standard 3-Tier-Architecture pattern:

\begin{enumerate}
      \item \textbf{Presentation Tier:} This is the upper layer that directly interacts with users. The presentation tier
            focuses on displaying information to the user and capturing user inputs. It's designed to handle all client-side
            interactions and renders the application's visual components.
      \item \textbf{Application Tier:} This layer contains the core business logic and processing rules of the
            application. It receives requests from the presentation tier, applies business rules, validates data, and
            coordinates the flow of information. This tier acts as an intermediary, ensuring that data is processed
            according to the application's specific requirements before being passed to or retrieved from the data tier.
      \item \textbf{Data Tier:} The bottom layer of the architecture, responsible for storing, retrieving, and managing
            data. The data tier ensures data integrity, provides database connection management, and implements data
            access methods that the application tier can use to interact with the stored information.
\end{enumerate}

The division of the system into these three layers allows a reduction in complexity, improves maintainability, and
facilitates scalability: each layer can be developed, tested, deployed and maintained independently, enabling a more
modular and flexible system architecture while also parallelizing development efforts.

The choice of a 3-Tier-Architecture is also motivated by the fact that it can be easily mapped to the MVC
(Model-View-Controller) pattern - a widely-used design pattern that separates the application into three interconnected
components: the Model (data), the View (interface), and the Controller (business logic). This pattern is particularly
useful since it allows for a clear separation of concerns and a more modular and maintainable codebase.

Each tier will be run inside a container environment (e.g. Docker) to ensure isolation and scalability. The containers
will be orchestrated (using adequate solutions e.g. Kubernetes) which will also manage the deployment and scaling of
the application.

\par{\textbf{Client-Server Communication:}} S\&C - like almost all modern web applications - is based on a
Client-Server architecture. The client (the user's browser) sends requests to the server, which processes them and
returns the appropriate responses. All the static content (HTML, CSS, JavaScript) is served directly using HTTPS, while
the dynamic content is generated by the server and sent back to the client as JSON data (REST over HTTPS) to be
rendered by the client-side JavaScript code.

\par{\textbf{Intra-Server Communication:}} The communication between the presentation tier and the application tier
will be based on gRPC over TCP/IP. gRPC is a high-performance, open-source, universal RPC framework that can run in any
environment allowing for an easy and efficient communication between the two tiers. The communication between the
application tier and the data tier will be based on SQL Queries over TCP/IP (as implemented by the database driver).

\section{Other Design Decisions}
\label{sec:other-design-decisions}%

\par{Helpers:} In \ref{sec:selected-architectural-styles-patterns} we said that the application tier is running inside a
container. While this is true it must be noted that the application is not fully monolithic. The application tier is
composed of:

\begin{itemize}
      \item \textbf{Core:} The core of the application, it contains the business logic and the processing rules. It is
            the part that communicates with the data tier and the presentation tier.
      \item \textbf{Email Service:} A service that sends emails to users. It is a separate service because it is not
            strictly related to the core of the application and it can be easily scaled independently. Communicates
            with the core using SMTP over TCP/IP. Core prepares the email and sends it to the email service which
            sends it to the user.
      \item \textbf{Recommendation Engine:} A service that provides internships recommendations to STs. Communicates
            with the core using gRPC over TCP/IP. Core asks the recommendation engine for recommendations and the
            recommendation engine provides them.
      \item \textbf{Suggestion Engine:} A service that proposes STs to COs (as described in the RASD). Communicates with
            the core using gRPC over TCP/IP like the recommendation engine.
\end{itemize}

It should be also remarked that the application tier communicates with (external to S\&C) the various UN's SSOs using
HTTPS.

\par{\textbf{Networking:}} The internal S\&C network will solely rely on IPv6. This choice is motivated by the fact that
IPv6 is cheaper to maintain, more secure, and more scalable than IPv4. The load balancer at the edge will provide IPv4
connectivity to the outside world for compatibility reasons but all internal communication will be done using IPv6.
This choice will also make easier a multi-cloud deployment in the future.
