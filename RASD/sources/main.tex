%------------------------------------------------------------------------------
%    POLITECNICO DI MILANO - THESIS TEMPLATE
%------------------------------------------------------------------------------
\documentclass{Configuration_Files/PoliMi3i_thesis}

%------------------------------------------------------------------------------
%	REQUIRED PACKAGES AND CONFIGURATIONS
%------------------------------------------------------------------------------

% CONFIGURATIONS
\usepackage{parskip} % For paragraph layout
\usepackage{setspace} % For using single or double spacing
\usepackage{emptypage} % To insert empty pages
\usepackage{multicol} % To write in multiple columns (executive summary)
\setlength\columnsep{15pt} % Column separation in executive summary
\setlength\parindent{0pt} % Indentation
\raggedbottom

% PACKAGES FOR TITLES
\usepackage{titlesec}
\titlespacing{\section}{0pt}{3.3ex}{2ex}
\titlespacing{\subsection}{0pt}{3.3ex}{1.65ex}
\titlespacing{\subsubsection}{0pt}{3.3ex}{1ex}
\usepackage{color}

% PACKAGES FOR LANGUAGE AND FONT
\usepackage[english]{babel} % The document is in English
\usepackage[utf8]{inputenc} % UTF8 encoding
\usepackage[T1]{fontenc} % Font encoding
\usepackage[11pt]{moresize} % Big fonts

% PACKAGES FOR IMAGES
\usepackage{graphicx}
\usepackage{transparent} % Enables transparent images
\usepackage{eso-pic} % For the background picture on the title page
\usepackage{subfig} % Numbered and caption subfigures using \subfloat
\usepackage{tikz} % A package for high-quality hand-made figures
\usetikzlibrary{}
\graphicspath{{./Images/}} % Directory of the images
\usepackage{caption} % Coloured captions
\usepackage{xcolor} % Coloured captions
\usepackage{amsthm,thmtools,xcolor} % Coloured "Theorem"
\usepackage{float}

% STANDARD MATH PACKAGES
\usepackage{amsmath}
\usepackage{amsthm}
\usepackage{amssymb}
\usepackage{amsfonts}
\usepackage{bm}
\usepackage[overload]{empheq} % For braced-style systems of equations
\usepackage{fix-cm} % To override original LaTeX restrictions on sizes

% PACKAGES FOR TABLES
\usepackage{tabularx}
\usepackage{longtable} % Tables that can span several pages
\usepackage{colortbl}

% PACKAGES FOR ALGORITHMS (PSEUDO-CODE)
\usepackage{algorithm}
\usepackage{algorithmic}

% PACKAGES FOR REFERENCES & BIBLIOGRAPHY
\usepackage[colorlinks=true, linkcolor=black, anchorcolor=black, citecolor=black, filecolor=black, menucolor=black, 
            runcolor=black, urlcolor=black]{hyperref} % Adds clickable links at references
\usepackage{cleveref}
\usepackage[square, numbers, sort&compress]{natbib} % Square brackets, citing references with numbers, citations sorted 
                                                    % by appearance in the text and compressed
\bibliographystyle{abbrvnat} % You may use a different style adapted to your field

% OTHER PACKAGES
\usepackage{pdfpages} % To include a pdf file
\usepackage{afterpage}
\usepackage{lipsum} % DUMMY PACKAGE
\usepackage{fancyhdr} % For the headers
\fancyhf{}

% CONFIGURATION FILE
% Define blue color typical of polimi
\definecolor{bluepoli}{cmyk}{0.4,0.1,0,0.4}

% Custom theorem environments
\declaretheoremstyle[
  headfont=\color{bluepoli}\normalfont\bfseries,
  bodyfont=\color{black}\normalfont\itshape,
]{colored}

% Set-up caption colors
\captionsetup[figure]{labelfont={color=bluepoli}} % Set colour of the captions
\captionsetup[table]{labelfont={color=bluepoli}} % Set colour of the captions
\captionsetup[algorithm]{labelfont={color=bluepoli}} % Set colour of the captions

\theoremstyle{colored}
\newtheorem{theorem}{Theorem}[chapter]
\newtheorem{proposition}{Proposition}[chapter]

% Enhances the features of the standard "table" and "tabular" environments.
\newcommand\T{\rule{0pt}{2.6ex}}
\newcommand\B{\rule[-1.2ex]{0pt}{0pt}}

% Pseudo-code algorithm descriptions.
\newcounter{algsubstate}
\renewcommand{\thealgsubstate}{\alph{algsubstate}}
\newenvironment{algsubstates}
{\setcounter{algsubstate}{0}%
  \renewcommand{\STATE}{%
    \stepcounter{algsubstate}%
    \Statex {\small\thealgsubstate:}\space}}
{}

% New font size
\newcommand\numfontsize{\@setfontsize\Huge{200}{60}}

% Title format: chapter
\titleformat{\chapter}[hang]{
  \fontsize{50}{20}\selectfont\bfseries\filright}{\textcolor{bluepoli} \thechapter\hsp\hspace{2mm}\textcolor{bluepoli}{|   }\hsp}{0pt}{\huge\bfseries \textcolor{bluepoli}
}

% Title format: section
\titleformat{\section}
{\color{bluepoli}\normalfont\Large\bfseries}
{\color{bluepoli}\thesection.}{1em}{}

% Title format: subsection
\titleformat{\subsection}
{\color{bluepoli}\normalfont\large\bfseries}
{\color{bluepoli}\thesubsection.}{1em}{}

% Title format: subsubsection
\titleformat{\subsubsection}
{\color{bluepoli}\normalfont\large\bfseries}
{\color{bluepoli}\thesubsubsection.}{1em}{}

% Shortening for setting no horizontal-spacing
\newcommand{\hsp}{\hspace{0pt}}

\makeatletter
% Renewcommand: cleardoublepage including the background pic
\renewcommand*\cleardoublepage{%
  \clearpage\if@twoside\ifodd\c@page\else
      \null
      \AddToShipoutPicture*{\BackgroundPic}
      \thispagestyle{empty}%
      \newpage
      \if@twocolumn\hbox{}\newpage\fi\fi\fi}
\makeatother

%For correctly numbering algorithms
\numberwithin{algorithm}{chapter}

%----------------------------------------------------------------------------
%	STARTING POINT OF THE DOCUMENT
%----------------------------------------------------------------------------
\begin{document}

\fancypagestyle{plain}{%
    \fancyhf{} % Clear all header and footer fields
    \fancyhead[RO,RE]{\thepage} % RO=right odd, RE=right even
    \renewcommand{\headrulewidth}{0pt}
    \renewcommand{\footrulewidth}{0pt}}

%----------------------------------------------------------------------------
%	TITLE PAGE
%----------------------------------------------------------------------------
\pagestyle{empty} % No page numbers
\frontmatter % Use roman page numbering style (i, ii, iii, iv...) for the preamble pages

\puttitle{
    documentTitle = Students\&Companies: \\ Requirements Analysis and \\ Specification Document (RASD),
    course = Software Engineering II,
    authorOne = Marco Apollonio (10764083),
    authorTwo = Giacomo Bossi (10766073),
    authorThree = Lorenzo Chiroli (10797603),
    advisor= Prof.ssa. Elisabetta Di Nitto,
    academicYear= 2024-2025
}

%----------------------------------------------------------------------------
%	LIST OF CONTENTS/FIGURES/TABLES/SYMBOLS
%----------------------------------------------------------------------------
\setcounter{page}{1} % Set page counter to 1
\thispagestyle{empty} % No page numbers on this page
\tableofcontents % Write out the Table of Contents
\thispagestyle{empty} % No page numbers on this page
\cleardoublepage % Start the document on a right-hand page

%-------------------------------------------------------------------------
%	DOCUMENT MAIN TEXT
%-------------------------------------------------------------------------
\addtocontents{toc}{\vspace{2em}} % Add a gap in the Contents, for aesthetics
\mainmatter % Begin numeric (1,2,3...) page numbering

\chapter{Introduction}
\label{ch:introduction}% 
% The \label{...}% enables to remove the small indentation that is generated, always leave the % symbol.

\par In today's rapidly evolving job market, there is a growing concern regarding the skill mismatch between fresh
graduates and the requirements of employers. This issue is particularly prevalent in the realm of internships,
especially in the STEM field, where students are often not able to meet the demands of industry-relevant roles, while
companies struggle to find candidates with the necessary skills for their internship programs.

\par This skill mismatch not only harms the career development of students but also affects the productivity of the
companies themselves. On one hand, students miss out on valuable learning opportunities and real-world experiences. On
the other hand, companies experience inefficiencies due to the time and resources spent on training interns who lack
the essential skills required for their roles.

\par To address this issue, we wish to establish a platform that connects students with relevant internship
opportunities while simultaneously providing employers with a pool of qualified candidates. S\&C also aims to empower
universities to take a proactive role in resolving potential issues that may arise between students and companies, thus
creating a safe space for both parties.

\section{Purpose}
\label{sec:purpose}%

\par This document serves mainly as a comprehensive reference for the development team involved in implementing
Students\&Companies.

\par This material also aims to provide an authoritative overview of the software's capabilities and constraints,
ensuring clear understanding among all stakeholders: students, companies, and university personnel.

\subsection{Goals}
\label{subsec:goals}%

\begin{longtable}{|l|p{0.9\textwidth}|}
    \hline
    \textbf{Goal} & \textbf{Description}                                                                                                                              \\
    \hline \hline
    \multicolumn{2}{|l|}{\textit{From a student perspective...}}                                                                                                      \\
    \hline
    G01           & Allow the ST to login using the UN-provided credentials.                                                                                          \\
    \hline
    G02           & Allow the ST to upload their own CV on the platform.                                                                                              \\
    \hline
    G03           & Allow the ST to see the internships published by the various COs.                                                                                 \\
    \hline
    G04           & Allow the ST to receive suggestions about internships that he might be interested in.                                                             \\
    \hline
    G05           & Allow the ST to apply for an internship.                                                                                                          \\
    \hline
    G06           & Allow the ST to fill out the interview questionnaires sent by the COs.                                                                            \\
    \hline
    G07           & Allow the ST to be contacted directly by the COs in order to start the internship (if they wish so).                                              \\
    \hline
    G08           & Allow the ST to complain if any problems occur during the internship itself.                                                                      \\
    \hline
    G09           & Allow the ST to give feedback after the internship is done.                                                                                       \\
    \hline
    G10           & Allow the ST to see a checklist with recommendations on how to make their CV better.                                                              \\
    \hline
    G11           & Allow the ST to be informed on the status of their own internships.                                                                               \\
    \hline \hline
    \multicolumn{2}{|l|}{\textit{From a company perspective...}}                                                                                                      \\
    \hline
    G12           & Allow the CO to login using the provided (after-sale) credentials.                                                                                \\
    \hline
    G13           & Allow the CO to edit their own CO profile (public, visible to STs and UNs).                                                                       \\
    \hline
    G14           & Allow the CO to publish an internship announcement.                                                                                               \\
    \hline
    G15           & Allow the CO to receive CV suggestions corresponding to its needs and asks these people to apply for the internships.                             \\
    \hline
    G16           & Allow the CO to see applications and selected them to proceed with the questionnaire compilations by the STs.                                     \\
    \hline
    G17           & Allow the CO to see the questionnaire results to choose STs to contact.                                                                           \\
    \hline
    G18           & Allow the CO to update the internship status (by also specifying who is taking to work with) by also adding comments if necessary during the job. \\
    \hline
    G19           & Allow the CO to complain if any problems occur during the internship itself.                                                                      \\
    \hline
    G20           & Allow the CO to see a checklist with recommendations on how to make their internship announcement better.                                         \\
    \hline
    G21           & Allow the CO to be informed on the status of their own internships.                                                                               \\
    \hline \hline
    \multicolumn{2}{|l|}{\textit{From a university perspective...}}                                                                                                   \\
    \hline
    G22           & Allow the UN to login using the provided (after-sale) credentials.                                                                                \\
    \hline
    G23           & Allow the UN to see the complaints and make proactive decisions about them.                                                                       \\
    \hline
    G24           & Allow the UN to fully monitor the status and outcome of all past and present internships.                                                         \\
    \hline
    \caption{S\&C Goals Table}
    \label{tab:goals}
\end{longtable}

\section{Scope}
\label{sec:scope}%

\par S\&C is a platform that matches students seeking internships with companies offering these opportunities. The
platform manages internship postings and student applications, while allowing companies to evaluate candidates through
customized questionnaires. Throughout the internship, both parties can provide feedback on its progress, with
universities mediating any disputes that arise.

\par The platform employs recommendation algorithms to suggest relevant internship opportunities to students based on
their skills, while helping companies discover potential candidates that match their requirements.

\subsection{World Phenomena}
\label{subsec:world-phenomena}%

\begin{longtable}{|l|p{0.8\textwidth}|}
    \hline
    \textbf{Phenomena} & \textbf{Description}                                                                \\
    \hline \hline
    WP01               & STs build their own CVs.                                                            \\
    \hline
    WP02               & COs organize and manage their own internships.                                      \\
    \hline
    WP03               & COs plan the questions for the questionnaire they want to submit to the STs.        \\
    \hline
    WP04               & COs make contact with the eligible STs according to the questionnaire results.      \\
    \hline
    WP05               & STs take part in the internship if they've been selected by the CO.                 \\
    \hline
    WP06               & The CO contacts the S\&C sales team if they want to be enrolled in the application. \\
    \hline
    WP07               & UNs contact the S\&C sales team if they want to be enrolled in the application.     \\
    \hline
    \caption{World Phenomena Table}
    \label{tab:world-phenomena}
\end{longtable}

\subsection{Shared Phenomena}
\label{subsec:shared-phenomena}%

\begin{longtable}{|l|p{0.5\textwidth}|l|l|}
    \hline
    \textbf{Phenomena} & \textbf{Description}                                                                                                                                                                                                                                  & \textbf{Controller} & \textbf{Observer} \\
    \hline \hline
    SP01               & STs upload or modify their CVs on the platform.                                                                                                                                                                                                       & ST                  & S\&C              \\
    \hline
    SP02               & Each CV uploaded to S\&C is saved and processed by the CV analyzer, which will perform a feature extraction analysis. After the inspection, the owner of the document can see some suggestions on how to improve it.                                  & S\&C                & ST                \\
    \hline
    SP03               & STs look for interesting internships on the platform.                                                                                                                                                                                                 & ST                  & S\&C              \\
    \hline
    SP04               & The machine informs the users about relevant internships by sending a notification via email.                                                                                                                                                         & S\&C                & ST                \\
    \hline
    SP05               & STs apply for internships on the platform.                                                                                                                                                                                                            & ST                  & S\&C              \\
    \hline
    SP06               & STs complete the questionnaire for an internship.                                                                                                                                                                                                     & ST                  & S\&C              \\
    \hline
    SP07               & The machine assesses the questionnaire responses according to specified CO-provided criteria.                                                                                                                                                         & S\&C                & CO                \\
    \hline
    SP08               & If any problems occur during the internship, the STs can send a feedback report.                                                                                                                                                                      & ST                  & S\&C              \\
    \hline
    SP09               & The platform will forward internship feedback to the UN: a notification will be sent via email, and the UN can review them on S\&C.                                                                                                                   & S\&C                & UN                \\
    \hline
    SP10               & The S\&C sales team enrolls a CO on the platform. A set of unique credentials is created.                                                                                                                                                             & S\&C                & CO                \\
    \hline
    SP11               & COs edit their public profiles on S\&C.                                                                                                                                                                                                               & CO                  & S\&C              \\
    \hline
    SP12               & COs publish internship ads on S\&C.                                                                                                                                                                                                                   & CO                  & S\&C              \\
    \hline
    SP13               & Each internship post is analyzed and indexed. After the analysis, the CO can see suggestions on how to improve the copy.                                                                                                                              & S\&C                & CO                \\
    \hline
    SP14               & The machine sends to the COs CVs of relevant STs (matching the requirements of published internship ads or keywords on the matching CO profile). The notification is sent via email, and the CV is visible in a section of the platform.              & S\&C                & CO                \\
    \hline
    SP15               & COs can request that STs who have applied to an internship to fill out the questionnaire.                                                                                                                                                             & CO                  & S\&C              \\
    \hline
    SP16               & COs update the internship status on S\&C during the progression (e.g., “waiting for applications”, “ongoing,”, “closed,”, “suspended,”, etc.).                                                                                                        & CO                  & S\&C              \\
    \hline
    SP17               & COs can send feedbacks on the internship if they require it.                                                                                                                                                                                          & CO                  & S\&C              \\
    \hline
    SP18               & The platform will forward internship feedback to the UN: a notification will be sent via email, and the UN can review them on S\&C.                                                                                                                   & S\&C                & UN                \\
    \hline
    SP19               & S\&C sales team enrolls a UN on S\&C. The organization's Active Directory system is linked to the platform. The machine contacts the UN's credentials directory, allowing STs and relevant personnel to log in using their institutional credentials. & S\&C                & UN                \\
    \hline
    SP20               & UNs respond to the feedbacks received on the platform by STs and/or COs.                                                                                                                                                                              & UN                  & S\&C              \\
    \hline
    SP21               & UNs see past and present internships on the platform and their status.                                                                                                                                                                                & UN                  & S\&C              \\
    \hline
    \caption{Shared Phenomena Table}
    \label{tab:shared-phenomena}
\end{longtable}

\section{Definitions, Acronyms, and Abbreviations}
\label{sec:definitions-acronyms-abbreviations}%

\begin{longtable}{|l|p{0.835\textwidth}|}
    \hline
    \textbf{Acronym} & \textbf{Definition}                                 \\
    \hline \hline
    RADS             & Requirements Analysis and Definition Specification. \\
    \hline
    STEM             & Science, Technology, Engineering, and Mathematics.  \\
    \hline
    S\&C             & Students\&Companies.                                \\
    \hline
    ST               & Student.                                            \\
    \hline
    CO               & Company.                                            \\
    \hline
    UN               & University.                                         \\
    \hline
    CV               & Curriculum Vitae.                                   \\
    \hline
    GOx              & Goal Number $x$.                                    \\
    \hline
    WPx              & World Phenomenon Number $x$.                        \\
    \hline
    SPx              & Shared Phenomenon Number $x$.                       \\
    \hline
    \caption{Definitions, Acronyms, and Abbreviations Table}
    \label{tab:definitions-acronyms-abbreviations}
\end{longtable}

\section{Revision History}
\label{sec:revision-history}%

\begin{itemize}
    %% FIXME: Update the revision history before the final release.
    \item \textbf{Version 1.0} (2024-10-30): Initial release.
\end{itemize}

\section{Reference Documents}
\label{sec:reference-documents}%

\begin{itemize}
    \item Assignment description.
    \item Provided course materials and notes.
\end{itemize}

\section{Document Structure}
\label{sec:document-structure}%

\begin{enumerate}
    \item \textbf{Introduction}: This section describes the purpose and goals of the project, highlighting its scope by
          analyzing relevant phenomena. Key terms, acronyms, and abbreviations will be defined for clarity, along with a
          revision history and a list of reference documents. The section concludes with an overview of the document’s
          structure.
    \item \textbf{Overall Description}: This section provides an overview of the product’s context, including
          scenarios, detailed domain models (such as class and state diagrams), and descriptions of key functions. It also
          tries to cover user characteristics (to better define stakeholder needs), along with any domain-specific
          assumptions, dependencies (external to S\&C itself), and constraints that will impact the software design and
          implementation.
    \item \textbf{Specific Requirements}: This section expands the general description by specifying external interface
          requirements, including user, hardware, software, and communication interfaces ("the world and the machine").
          Functional requirements are defined with use case diagrams and activity diagrams. Performance targets, design
          constraints, and software attributes— reliability, availability, security, and maintainability—are detailed.
    \item \textbf{Formal Analysis}: This section describes the main objectives of the formal modeling activity and the
          structure of the Alloy model. Key assertions and results of checks will try to demonstrate the model’s accuracy, a
          focus will also put on why these result matters for the project.
    \item \textbf{Effort Spent}: This section documents the hours worked by each team member.
    \item \textbf{References}: This section collects all sources cited within the document.
\end{enumerate}


\end{document}